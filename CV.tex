\documentclass[11pt, letterpaper]{article}

% Packages:
\usepackage[
    ignoreheadfoot, % set margins without considering header and footer
    top=2 cm, % seperation between body and page edge from the top
    bottom=2 cm, % seperation between body and page edge from the bottom
    left=1.5 cm, % seperation between body and page edge from the left
    right=1.5 cm, % seperation between body and page edge from the right
    footskip=1.0 cm, % seperation between body and footer
    % showframe % for debugging 
]{geometry} % for adjusting page geometry
\usepackage{titlesec} % for customizing section titles
\usepackage{tabularx} % for making tables with fixed width columns
\usepackage{array} % tabularx requires this
\usepackage[dvipsnames]{xcolor} % for coloring text
\definecolor{primaryColor}{RGB}{0, 0, 0} % define primary color
\usepackage{enumitem} % for customizing lists
\usepackage{fontawesome5} % for using icons
\usepackage{amsmath} % for math
\usepackage[
    pdftitle={Martino Piaggi's CV},
    pdfauthor={Martino Piaggi},
    pdfcreator={LaTeX with RenderCV},
    colorlinks=true,
    urlcolor=primaryColor
]{hyperref} % for links, metadata and bookmarks
\usepackage[pscoord]{eso-pic} % for floating text on the page
\usepackage{calc} % for calculating lengths
\usepackage{bookmark} % for bookmarks
\usepackage{lastpage} % for getting the total number of pages
\usepackage{changepage} % for one column entries (adjustwidth environment)
\usepackage{paracol} % for two and three column entries
\usepackage{ifthen} % for conditional statements
\usepackage{needspace} % for avoiding page brake right after the section title
\usepackage{iftex} % check if engine is pdflatex, xetex or luatex

% Ensure that generate pdf is machine readable/ATS parsable:
\ifPDFTeX
    \input{glyphtounicode}
    \pdfgentounicode=1
    \usepackage[T1]{fontenc}
    \usepackage[utf8]{inputenc}
    \usepackage{lmodern}
\fi

\usepackage{charter}

% Some settings:
\raggedright
\AtBeginEnvironment{adjustwidth}{\partopsep0pt} % remove space before adjustwidth environment
\pagestyle{empty} % no header or footer
\setcounter{secnumdepth}{0} % no section numbering
\setlength{\parindent}{0pt} % no indentation
\setlength{\topskip}{0pt} % no top skip
\setlength{\columnsep}{0.15cm} % set column seperation
\pagenumbering{gobble} % no page numbering

\titleformat{\section}{\needspace{4\baselineskip}\bfseries\large}{}{0pt}{}[\vspace{1pt}\titlerule]

\titlespacing{\section}{
    % left space:
    -1pt
}{
    % top space:
    0.1 cm
}{
    % bottom space:
    0.1 cm
} % section title spacing

\renewcommand\labelitemi{$\vcenter{\hbox{\small$\bullet$}}$} % custom bullet points
\newenvironment{highlights}{
    \begin{itemize}[
        topsep=0.10 cm,
        parsep=0.10 cm,
        partopsep=0pt,
        itemsep=0pt,
        leftmargin=0 cm + 10pt
    ]
}{
    \end{itemize}
} % new environment for highlights

\newenvironment{highlightsforbulletentries}{
    \begin{itemize}[
        topsep=0.10 cm,
        parsep=0.10 cm,
        partopsep=0pt,
        itemsep=0pt,
        leftmargin=10pt
    ]
}{
    \end{itemize}
} % new environment for highlights for bullet entries

\newenvironment{onecolentry}{
    \begin{adjustwidth}{
        0 cm + 0.00001 cm
    }{
        0 cm + 0.00001 cm
    }
}{
    \end{adjustwidth}
} % new environment for one column entries

\newenvironment{twocolentry}[2][]{
    \onecolentry
    \def\secondColumn{#2}
    \setcolumnwidth{\fill, 4.5 cm}
    \begin{paracol}{2}
}{
    \switchcolumn \raggedleft \secondColumn
    \end{paracol}
    \endonecolentry
} % new environment for two column entries

\newenvironment{threecolentry}[3][]{
    \onecolentry
    \def\thirdColumn{#3}
    \setcolumnwidth{, \fill, 4.5 cm}
    \begin{paracol}{3}
    {\raggedright #2} \switchcolumn
}{
    \switchcolumn \raggedleft \thirdColumn
    \end{paracol}
    \endonecolentry
} % new environment for three column entries

\newenvironment{header}{
    \setlength{\topsep}{0pt}\par\kern\topsep\centering\linespread{1.5}
}{
    \par\kern\topsep
} % new environment for the header

\newcommand{\placelastupdatedtext}{% \placetextbox{<horizontal pos>}{<vertical pos>}{<stuff>}
  \AddToShipoutPictureFG*{% Add <stuff> to current page foreground
    \put(
        \LenToUnit{\paperwidth-2 cm-0 cm+0.05cm},
        \LenToUnit{\paperheight-1.0 cm}
    ){\vtop{{\null}\makebox[0pt][c]{
        \small\color{gray}\textit{Last updated in July 2024}\hspace{\widthof{Last updated in July 2024}}
    }}}%
  }%
}%

% save the original href command in a new command:
\let\hrefWithoutArrow\href

\begin{document}
    \newcommand{\AND}{\unskip
        \cleaders\copy\ANDbox\hskip\wd\ANDbox
        \ignorespaces
    }
    \newsavebox\ANDbox
    \sbox\ANDbox{$|$}

    \begin{header}
        \fontsize{25 pt}{25 pt}\selectfont Martino Piaggi

        \vspace{5 pt}

        \normalsize
        \mbox{20 August 2000}%
        \kern 5.0 pt%
        \AND%
        \mbox{\hrefWithoutArrow{https://github.com/martinopiaggi}{github.com/martinopiaggi}}%
        \AND%
         \kern  5.0 pt%
        \mbox{\hrefWithoutArrow{mailto:martino.piaggi@pm.me}{martino.piaggi@pm.me}}%
        \AND%
        \kern 5.0 pt%
        \mbox{\hrefWithoutArrow{https://martino.im}{martino.im}}%
        \kern 5.0 pt%
        \AND%
        \kern 5.0 pt%
        \mbox{\hrefWithoutArrow{https://linkedin.com/in/martinopiaggi}{linkedin.com/in/martinopiaggi}}%
        \kern 5.0 pt%

    \end{header}

    \vspace{5 pt - 0.3 cm}

    \section{Education}

    \begin{twocolentry}{
        2022 – Present
    }
        \textbf{Politecnico di Milano}, Master's degree in Computer Science Engineering
    \end{twocolentry}
    \vspace{0.10 cm}
    \begin{onecolentry}
        \begin{highlights}
            \item \text{Relevant coursework:} Videogame Programming and Design, Advanced Algorithms and Parallel Programming, Distributed Systems, Computer Graphics
        \end{highlights}
    \end{onecolentry}
    \vspace{0.2 cm}

    \begin{twocolentry}{
        2019 – 2022
    }
        \textbf{Politecnico di Milano}, Bachelor of Engineering in Computer Science Engineering
    \end{twocolentry}

    \vspace{0.10 cm}
    \begin{onecolentry}
        \begin{highlights}
            \item \text{Relevant coursework:} Algorithms and Data Structures, Foundations of AI, Software Engineering, Geometry and Linear Algebra, Physics and Rational Mechanics
        \end{highlights}
    \end{onecolentry}



\section{Projects}

\vspace{0.10 cm}
\begin{onecolentry}
    \begin{highlights}
        \item \textbf{MSc Thesis:} (In progress) Procedural Content Generation for racing tracks using Quality-Diversity (QD) Reinforcement Learning algorithm. \newline \textit{Technologies:} Python, JavaScript, Docker, C++, Unreal Engine
        \item \textbf{\href{https://github.com/martinopiaggi/gambetto}{Gambetto}:} Led a team of five engineering students in developing a game for the Videogame Design course. Achieved a score of 30/30L based on professors and peer reviews. 
        \newline \textit{Technologies:} Unity, C\#
        \item \textbf{\href{https://github.com/martinopiaggi/ShutTheBox}{Shut The Box}:} Implemented an AI-powered version of a classic game using Monte Carlo Tree Search algorithm from scratch.
        \newline \textit{Technologies:} Unity, C\#
        \item \textbf{\href{https://github.com/martinopiaggi/Unity-Maze-generation-using-disjoint-sets}{Unity Maze Generation using Disjoint Sets}:} Efficient procedural maze generation algorithm.
        \newline \textit{Technologies:} Unity, C\#
        \item 
        \textbf{\href{https://github.com/singhamrit99/ing-sw-2022-Piaggi-Perini-Singh}{Java board game}:} A complex table-top game built with a team of three people for Software Engineering course project. It included online multiplayer. Achieved maximum grade. 
         \newline \textit{Technologies:} Java, Java RMI, JavaFX
        \item \textbf{\href{https://github.com/lorenzo-morelli/The-Pirate-Bay}{The Pirate Bay}:} An exploration of the Vulkan graphics API. \newline \textit{Technologies:} C++, Vulkan API, GLSL
        \item \textbf{\href{https://github.com/martinopiaggi/Itch.io-mobile-app}{Unofficial Itch.io Mobile App}:} Mobile interface for the indie game platform, with additional full-stack features.  \newline \textit{Technologies:} Flutter, Firebase
    \end{highlights}
\end{onecolentry}

    \section{Experience}

    \begin{twocolentry}{
        2018 – 2024
    }
        \textbf{Tech Artist}, Fiverr (Freelance part-time)
    \end{twocolentry}

    \vspace{0.10 cm}
    \begin{onecolentry}
        \begin{highlights}
            \item 3D art and rendering animation using Unreal Engine, Cinema 4D and Adobe Suite for indie creators and small businesses.
            \item 4.8/5 star rating across 68 public reviews.
        \end{highlights}
    \end{onecolentry}

    \vspace{0.2 cm}

    \begin{twocolentry}{
        2020 – 2021
    }
        \textbf{Core Team Member}, Google Developer Student Club PoliMi
    \end{twocolentry}

    \vspace{0.10 cm}
    \begin{onecolentry}
        \begin{highlights}
            \item  I was selected as team member Google Developer Student Club community at Politecnico di Milano. 
            \item Organized tech workshops with different companies (speakers from Google, Waymo, Avalanche and others) to a community of 1200+ engineering students.
        \end{highlights}
    \end{onecolentry}

  
    \section{Skills \& Technologies}

    \begin{onecolentry}
        \textbf{Programming:} C\#, C++, Java, JavaScript,  Python, SQL, academic experience with Erlang, Scheme, Haskell
    \end{onecolentry}

    \vspace{0.1 cm}

    \begin{onecolentry}
        \textbf{Frameworks:} Unity, Unreal Engine, Three.js, Google Firebase, Flutter, academic experience with Vulkan and CUDA
    \end{onecolentry}

\end{document}