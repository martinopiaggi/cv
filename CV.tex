\documentclass[10pt, letterpaper]{article}

% Essential packages
\usepackage[ignoreheadfoot, top=1.5cm, bottom=1.5cm, left=1.5cm, right=1.5cm]{geometry}
\usepackage{titlesec}
\usepackage{tabularx, array}
\usepackage{inconsolata}
\renewcommand*\familydefault{\ttdefault}
\usepackage[T1]{fontenc}
\usepackage[dvipsnames]{xcolor}
\definecolor{urlcolor}{RGB}{128, 85, 227}

\usepackage{enumitem}
\usepackage{amsmath}
\usepackage[
    pdftitle={Martino Piaggi - Game Programmer (Unreal Engine) CV},
    pdfauthor={Martino Piaggi},
    pdfsubject={Game Programming, Unreal Engine, Gameplay Programming},
    pdfkeywords={Martino Piaggi, Game Programmer, Gameplay Programmer, Unreal Engine, C++, C\#, Unity},
    colorlinks=true,
    urlcolor=urlcolor,
    linkcolor=urlcolor
]{hyperref}
\usepackage{eso-pic, calc}
\usepackage{changepage}
\usepackage{needspace}

% Basic settings
\raggedright
\pagestyle{empty}
\setcounter{secnumdepth}{0}
\setlength{\parindent}{0pt}
\pagenumbering{gobble}

% Section formatting
\titleformat{\section}{\needspace{3\baselineskip}\bfseries\large}{}{0pt}{}[\vspace{1pt}\titlerule]
\titlespacing{\section}{0pt}{0.4cm}{0.2cm}

% List environments
\renewcommand\labelitemi{$\vcenter{\hbox{\small$\bullet$}}$}
\newenvironment{highlights}{
    \begin{itemize}[topsep=0cm, parsep=0.05cm, itemsep=0.05cm, leftmargin=1.0cm]
}{
    \end{itemize}
}

\begin{document}

% Header
\begin{center}
    \fontsize{26pt}{26pt}\selectfont\textbf{Martino Piaggi}

    \vspace{0.3cm}
    \normalsize
    \href{https://github.com/martinopiaggi}{github.com/martinopiaggi} $\bullet$
    \href{https://martino.im}{martino.im} $\bullet$ 
    \href{mailto:martino.piaggi@pm.me}{martino.piaggi@pm.me}  
\end{center}

\vspace{0.2cm}

\section{Experience}

\begin{tabularx}{\textwidth}{@{}Xr@{}}
    \textbf{Junior Game Programmer}, \href{https://games.milestone.it/}{Milestone} & \textbf{Nov 2024 – Present}
\end{tabularx}

\begin{highlights}
\item Implemented core gameplay features for an upcoming racing title 
\item Collaborated with senior engineers and designers to define requirements and develop robust and reusable C++ systems within Unreal Engine, adhering to object-oriented programming principles and industry best practices
\end{highlights}

\vspace{0.2cm}

\begin{tabularx}{\textwidth}{@{}Xr@{}}
    \textbf{Tech Artist}, \href{https://www.fiverr.com/sbdsurface}{Fiverr}  & \textbf{2018 – 2024}
\end{tabularx}

\begin{highlights}
    \item Created 3D illustrations and animations using Unreal Engine, Cinema 4D and Adobe Suite for indie creators and small businesses 
    \item Maintained 4.8/5 star rating across 68 public reviews as part-time freelancer alongside university coursework
\end{highlights}

\vspace{0.2cm}

\section{Education}

\begin{tabularx}{\textwidth}{@{}Xr@{}}
    \textbf{Politecnico di Milano}, Master's degree in Computer Science Engineering & \textbf{Sep 2022 – 2025}
\end{tabularx}

\begin{highlights}
    \item \textbf{Relevant coursework:} Videogame Programming and Design, Advanced Algorithms and Parallel Programming, Distributed Systems, Computer Graphics
\end{highlights}

\vspace{0.2cm}

\begin{tabularx}{\textwidth}{@{}Xr@{}}
    \textbf{Politecnico di Milano}, Bachelor of Engineering in Computer Science Engineering & \textbf{2019 – 2022}
\end{tabularx}

\begin{highlights}
    \item \textbf{Relevant coursework:} Algorithms and Data Structures, Foundations of AI, Software Engineering, Geometry and Linear Algebra, Physics and Rational Mechanics
\end{highlights}

\vspace{0.2cm}

\section{Key Projects}

\begin{tabularx}{\textwidth}{@{}Xr@{}}
    \textbf{PCG for Racing Tracks}, MSc Thesis & \textbf{2024 – 2025}
\end{tabularx}

\begin{highlights}
    \item Developed procedural racing track generation system using evolutionary Quality-Diversity algorithms in \textbf{Python}, \textbf{C++}, \textbf{Docker}, and deployed interactive visualization with \textbf{JavaScript}.
\end{highlights}

\vspace{0.15cm}

\begin{tabularx}{\textwidth}{@{}Xr@{}}
    \textbf{\href{https://github.com/martinopiaggi/gambetto}{Gambetto}}, Course Project & \textbf{2023}
\end{tabularx}

\begin{highlights}
    \item Led a five-person team to deliver a polished game prototype in \textbf{Unity 3D (C\#)} and then open sourced
\end{highlights}

\vspace{0.15cm}

\begin{tabularx}{\textwidth}{@{}Xr@{}}
    \textbf{\href{https://github.com/martinopiaggi/ShutTheBox}{Shut The Box}}, Course Project & \textbf{2022}
\end{tabularx}

\begin{highlights}
    \item Implemented an AI-powered version of a puzzle game using Monte Carlo Tree Search algorithm in \textbf{Unity 3D (C\#)}
\end{highlights}

\vspace{0.15cm}

\begin{tabularx}{\textwidth}{@{}Xr@{}}
    \textbf{\href{https://github.com/martinopiaggi/Unity-Maze-generation-using-disjoint-sets}{Unity Maze Generation}}, Personal Project & \textbf{2021}
\end{tabularx}

\begin{highlights}
    \item Built an efficient procedural maze generator using disjoint sets data structure in \textbf{Unity 3D (C\#)}.
\end{highlights}


\section{Skills \& Technologies}

\begin{highlights}
    \item \textbf{Programming Languages:} C++, C\#, JavaScript, Python
    \item \textbf{Game Development:} Unreal Engine, Unity 3D
    \item \textbf{AI \& Generative Tech:} Stable Diffusion, Flux Models, LLMs, Prompt Engineering
    \item \textbf{3D Modeling:} Cinema 4D, Asset Optimization
\end{highlights}

\end{document}